\documentclass[11pt]{article}

\usepackage[english]{babel}
\usepackage[utf8]{inputenc}
\usepackage{fancyhdr}

\usepackage{hyperref}
\hypersetup{
    colorlinks=true,
    linkcolor=blue,
    filecolor=magenta,
    urlcolor=cyan,
}

\pagestyle{fancy}
\fancyhf{}

\lhead{CMPUT497-B2}
\rhead{Joseph Meleshko}

\begin{document}

\section*{Progress Report: Dark Hex Project}

At this point, I have finished the framework for interacting with Dark Hex.
My code so far can be found on \href{https://github.com/josephmeleshko/CMPUT497-Dark-Hex-Project}{GitHub}.
I have a working server written in python that manages the game and two trivial players to interact with it.
Since Dark Hex is an imperfect information game, a server to moderate a connection needs to handle the information both players have.
My server keeps track of a board state and also what each player knows.
When it is time for a player to move, the server builds an array representing the board as the player knows it,
keeping hidden the opponent's moves that they haven't discovered yet.
If the player makes a move that collides with the opponent, the server updates the player's knowledge and then queries them for a new move.
It provides the players with information on how the game is progressing,
alerting them if they collided with the opponent or succeeded and the opponent has made a move.
It also keeps time for each player, notifying them how much time they have used on each turn,
tracking total time that it takes for them to make a move that doesn't collide with the opponent.
It currently doesn't punish the players for exceeding the time, but it is prepared to enforce it if needed.
Also included are the two trivial players, a player that just makes a random legal move, and a player that queries the standard input for a move.
This means that as of now, a human can play a game against a random player.
To play against the random player, run "python3 DarkHexMain.py", to see two random players uncomment line 9 and comment out line 6 and run the same command.
The next steps are to implement an proper agent for the game, though I've currently been too preoccupied to start on it as of yet.

\subsection*{Github}
https://github.com/josephmeleshko/CMPUT497-Dark-Hex-Project
\end{document}
