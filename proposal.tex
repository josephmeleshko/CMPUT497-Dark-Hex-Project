\documentclass[11pt]{article}

\usepackage[english]{babel}
\usepackage[utf8]{inputenc}
\usepackage{fancyhdr}

\usepackage{hyperref}
\hypersetup{
    colorlinks=true,
    linkcolor=blue,
    filecolor=magenta,
    urlcolor=cyan,
}

\pagestyle{fancy}
\fancyhf{}

\lhead{CMPUT497-B2}
\rhead{Joseph Meleshko}

\begin{document}

\section*{Project Proposal: Dark Hex}

\subsection*{Dark Hex}

\noindent Dark Hex, (or Kriegspiel Hex, Phantom Hex) is a variant of the game Hex in which the players’ pieces are not made visible to their opponents until there is a collision in placement.
Dark Hex, is similar to its parent, Hex, in that it is a deterministic, finite, and alternate turn game.
It differs though, turning a perfect information game into an imperfect but complete information game.
I chose Dark Hex since I've already had some experience working with the game, and I found it extremely interesting.
As I understand there aren't many projects based on playing Dark Hex so a player would be a novel endeavor.

\subsection*{Project}

\noindent I intend to implement a simple visualizer/interface and a player.
I will start by implementing a simple protocol for communication between players, based on HTP/GTP. (Go Text Protocol/Hex Text Protocol)
Those protocols work for Hex and Go respectively, but being an imperfect information game and sometimes requiring multiple queries to a player per move will require a modified protocol.
Afterwards, a simple text visualizer in the terminal to begin testing and interacting with the player.
Time permitting, I may implement a better visualizer with a simple graphical interface rather than the text based terminal interface.
The main component, will be a player for Dark Hex.
My initial plan is to make a player for small Dark Hex boards, and see how it preforms and what improvements could be made to it.
A few improvements would be attempting to integrate some tools/algorithms from regular Hex such as dead cell pruning and others into a player for Dark Hex.
Working out a simple endgame solver to find forced wins would be a boon and also non-trivial given the imperfect information nature of Dark Hex.
Most of the code specific to Dark Hex will be my own, however some utilities from regular Hex playing may be used in attempting to make the player stronger.
I am the sole contributor to this project, so I will be responsible for all of the work.
I don't know how much time this project will take to complete, and the more time given will likely result in increasing the strength of the player.

\newpage
\subsection*{Resources}

\noindent \href{https://www.hexwiki.net/index.php/Main_Page}{Hex Wiki} is a resource with general information about Hex, which likely will be useful in building a Dark Hex player.\\
\noindent The \href{http://webdocs.cs.ualberta.ca/~hayward/hex/}{University Of Alberta Computer Hex Group} has many useful discoveries and utilities for Hex, several of which will likely be at least somewhat applicable to playing Dark Hex, pruning dead cells from search for instance.

\end{document}
